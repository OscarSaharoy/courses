%! TeX program = lualatex
\documentclass[11pt]{report}

\usepackage{graphicx}
\graphicspath{ {./figures} }

\usepackage{amsmath}
\usepackage{amsthm}
\usepackage{amssymb}
\usepackage{enumitem}

\usepackage{titlesec}
\titleformat{\chapter}
{\centering\sc\LARGE}{\thechapter.}{1em}{\Huge}
\titlespacing*{\chapter}{0cm}{0cm}{0.5cm}

\titleformat{\section}
{\sc}{\thesection}{1em}{\Large}

\titleformat{\subsection}
{\sc}{\thesubsection}{1em}{\large}

% Figure related
\usepackage{tikz}
\usetikzlibrary{calc}

% Page Geometry
\usepackage{geometry}
\geometry{a4paper, margin=2cm}

% Framing
\usepackage[framemethod=TikZ]{mdframed}

% Theorems
\usepackage{thmtools}
\declaretheoremstyle[headfont=\sc, bodyfont=\normalfont, mdframed={ needspace=4\baselineskip, nobreak=false } ]{fullbox}
\declaretheoremstyle[headfont=\sc, bodyfont=\normalfont, mdframed={ rightline=false, topline=false, bottomline=false, needspace=4\baselineskip, nobreak=false }]{sideline}

\declaretheorem[numberwithin=chapter, style=fullbox, name=Definition]{definition}
\declaretheorem[sibling=definition, style=fullbox, name=Corollary ]{corollary}
\declaretheorem[sibling=definition, style=fullbox, name=Lemma ]{lemma}
\declaretheorem[sibling=definition, style=fullbox, name=Proposition ]{proposition}
\declaretheorem[sibling=definition, style=fullbox, name=Theorem ]{theorem}
\declaretheorem[sibling=definition, style=sideline, name=Example ]{example}
\declaretheorem[sibling=definition, style=sideline, name=Nonexample ]{nonexample}
\declaretheorem[style=sideline, numbered=no, name=Remark ]{remark}
\declaretheorem[style=sideline, numbered=no, name=Notation ]{notation}
\declaretheorem[numberwithin=chapter, style=fullbox, name=Problem]{problem}
\declaretheorem[numberwithin=chapter, style=sideline, name=Solution]{solution}

%Exercise environment to be picked up by converter.
\newenvironment{exercise}{}{}

%Content classification macros.
\newcommand{\basic}{(Basic)}
\newcommand{\intermediate}{(Intermediate)}
\newcommand{\challenging}{(Challenging)}

%Chapter outline env
\newenvironment{chout}{\begin{center}\it}{\end{center}}

\setlength{\parskip}{0.3\baselineskip}
\setlength{\marginparpush}{0.5cm}
\setlength{\parindent}{0pt}

%Commands
\renewcommand{\vec}[1]{\textbf{#1}}
\newcommand{\defined}[1]{\textit{#1}}

%Operators
\let\diff\relax
\DeclareMathOperator{\diff}{\, d\!}

