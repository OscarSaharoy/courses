\chapter{Logic}

The field of logic is at the heart of mathematics, and essentially forms the language with which we can make mathematical statements. Logic is a deep, and well-explored area -- we will explore only the beginnings of the subject.

\section{Propositional logic}
Propositional logic is concerned with the relationships and connections between statements which are either true or false. If we have two statements $ A $ and $ B $, we will consider how the respective truth of each of these statements influence the truth of statements such as,
\begin{align*}
	A \text{ and } B, \quad A \text{ or } B, \quad \text{ not } A, \quad \text{ if } A \text{ then } B
\end{align*}

\begin{example}
	Suppose we have statement $ A $ as `I wake up late', and statement $ B $ as `I miss my train'. Think about how these statements are related to eachother. For instance, we could consider,
	\begin{align*}
		\text{ if } A \text{ then } B
	\end{align*}
	or, substituting for our plain english versions,
	\begin{align*}
		\text{ if I wake up late then I miss my train }
	\end{align*}
	which seems to make sense.

	It is also interesting to consider,
	\begin{align*}
		\text{ if } B \text{ then } A
	\end{align*}
	or,
	\begin{align*}
		\text{ if I miss my train then I woke up late }
	\end{align*}
	making a small alteration in the statement to account for the english past tense. Is this statement true?
\end{example}

We will be concerned mostly with three relationships between statements, and it is useful to map out these relationships in what we call a truth table. Throughout the following subsections, we will be considering two statements $ A $ and $ B $. In order to keep the tables readable, we will now label true by $ T $, and false by $ F $.

\subsection{Conjunction}
We can, as seen, create a compound statement from statements $ A $ and $ B $ as,
\begin{align*}
	A \text{ and } B
\end{align*}
which is a process we sometimes refer to as conjunction.

\begin{notation}
	It is common in logic to denote the operation of statement conjuction by the operator $ \land $, i.e., $ A \land B $, although we will do fine with `and'.
\end{notation}

Suppose I make a small change to the document, including some new maths,
\begin{align*}
	\sum_{n=1}^{\infty}{\frac{1}{n^{2}}} = \frac{\pi^{2}}{6}
\end{align*}
